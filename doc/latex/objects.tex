\section{Object Oriented Programming}
\markright{\arabic{section}. Object-Oriented Programming}

The structures and behaviors of objects are described in
classes, which are defined by
{\bf defclass} macro and {\bf defmethod} special form.
{\bf defclass} defines the name of the class, its super class, and
slot variable names, optionally with their types and message forwarding.
{\bf defmethod} defines methods which will invoked when
corresponding messages are sent.
Class definition is assigned to the symbol's special value.
You may think of {\bf class} as 
the counter part of Common Lisp's {\bf structure}.
Slot accessing functions and {\bf setf} methods are automatically defined
for each slot by {\bf defclass}.

Most classes are instantiated from the built-in class {\bf metaclass}.
Class {\bf vector-class}, which is a subclass of {\bf metaclass},
is a metaclass for vectors.
If you need to use class-variables and class-methods,
you may make your own metaclass by subclassing {\bf metaclass},
and the metaclass name should be given to {\bf defclass} with
{\tt :metaclass} keyword.

Vectors are different from other record-like objects
because an instance of the vector can have arbitrary number of elements,
while record-like objects have fixed number of slots.
EusLisp's object is either a record-like object or a vector, not both
at the same time.

Vectors whose elements are typed or the number of elements are
unchangeable can also be defined by {\bf defclass}.
In the following example,
class {\tt intvec5} which has five integer elements is defined.
Automatic type check and conversion are performed when the elements are
accessed by the interpreter.
When compiled with proper declaration, faster accessing code is produced.

\begin{verbatim}
(defclass intvec5 :super vector :element-type :integer :size 5)
(setq x (instantiate intvec5))  --> #i(0 0 0 0 0)
\end{verbatim}

When a message is sent to an object, 
the corresponding method is searched for, first in its class,
and next in its superclasses toward {\bf object},
until all superclasses are exhausted.
If the method is undefined, forward list is searched.
This forwarding mechanism is introduced to simulate multiple inheritance.
If the search fails again, a method named {\tt :nomethod} is searched,
and the method is invoked with a list of all the arguments.
In the following example, the messages {\tt :telephone} and {\tt :mail} are
sent to {\tt secretary} slot object which is typed {\tt person},
and {\tt :go-home} message is sent to {\tt chauffeur} slot.

\begin{verbatim}
(defclass president :super object
                    :slots ((name :type string)
                            (age  :type :integer)
                            (secretary  :type person
                                        :forward (:telephone :mail))
                            (chauffeur  :forward (:go-home))))
\end{verbatim}

In a method, two more local variables,
{\bf class} and {\bf self}, become accessible.
You should not change either of these variables.
If you do that, the ones supplied by the system are hidden,
and {\bf send-super} and {\bf send self} are confused.


\subsection{Classes and Methods}

\begin{refdesc}

\longdescription{defclass}
{classname \&key \= :super object \` [macro] \\
\> :slots (\{var $|$ (var [:type type] [:forward selectors])\}*) \\
\> :metaclass metaclass \\
\> :element-type t \\
\> :size -1\\}
{creates or redefine a class.
When a class is redefined to have different superclass or slot variables,
old objects instantiated from the previous class definition will 
behave unexpectedly, since method definitions assume the new slots
disposition.}

\specialdesc{defmethod}{classname \{(selector lambda-list . body)\}*}{
defines one or more methods of {\em classname}. Each {\em selector}
must be a keyword symbol.}

\macdesc{defclassmethod}{classname \{(selector lambda-list . body)\}*}

\funcdesc{classp}{object}{T if {\em object} is a class object, that is,
an instance of class {\bf metaclass} or its subclasses.}

\funcdesc{subclassp}{class super}{
Checks {\em class} is a subclass of {\em super}.}

\funcdesc{vector-class-p}{x}{T if {\em x} is an instance of {\bf vector-class}.}

\funcdesc{delete-method}{class method-name}{
The method definition is removed from the specified class.}

\funcdesc{class-hierarchy}{class}{
prints inheritance hierarchy below {\em class}.}

\funcdesc{system:list-all-classes}{}{
lists up all the classes defined so far.}

\funcdesc{system:find-method}{object selector}{
tries to find a method specified by {\em selector} in the class of {\em object}
and in its superclass.
This is used to know whether {\em object} can respond to {\em selector}.}

\funcdesc{system:method-cache}{[flag]}{
Interrogates the hit ratio of the method cache,
and returns a list of two numbers, hit and miss.
If {\em flag} is NIL, method caching is disabled.
If non-nil flag is given, method cache is purged and caching is enabled.}

\end{refdesc}

\subsection{Message Sending}
\begin{refdesc}

\funcdesc{send}{object selector \{arg\}*}{
send a message consisting of {\em selector} and {\em arg} to {\em object}.
{\em object} can be anything but number.
{\em selector} must be evaluated to be a keyword.}

\funcdesc{send-message}{target search selector \{arg\}*}{
Low level primitive to implement {\bf send-super}.}

\macrodesc{send*}{object selector \&rest msg-list}{
{\bf send*} applies {\bf send-message} to a list of arguments.
The relation between {\bf send} and {\bf send*} is like the one
between {\bf funcall} and {\bf apply}, or {\bf list} and {\bf list*}.}

\funcdesc{send-all}{receivers selector \&rest mesg}{
sends the same message to all the receivers, and collects the result
in a list.}

\macrodesc{send-super}{selector \&rest msgs}{
sends {\em msgs} to self, but
begins method searching at the superclass of the class
where the method currently being executed is defined.
It is an error to {\em send-super} outside a method (i.e. in a function).}

\macrodesc{send-super*}{selector \&rest msg-list}{
{\bf send-super*} is apply version of send-super.}

\end{refdesc}

\subsection{Instance Management}

\begin{refdesc}

\funcdesc{instantiate}{class \&optional size}{
the lowest primitive to create a new object from a class.
If the class is a vector-class, {\em size} should be supplied.}

\macrodesc{instance}{class \&rest message}{
An instance is created, and the message is sent to it.}

\funcdesc{make-instance}{class \&rest var-val-pairs}{
creates an instance of {\em class} and sets its slot variables
according to {\em var-val-pairs}.
For example, {\tt (make-instance cons :car 1 :cdr 2)}
is equivalent to {\tt (cons 1 2)}.}

\funcdesc{copy-object}{object}{
{\bf copy-object} function is used to copy objects keeping the referencing
topologies even they have recursive references.
{\bf Copy-object} copies any objects accessible from {\em object}
except symbols and packages, which are untouched to keep the uniqueness
of symbols.
{\bf copy-object} traverses all the references in an object twice:
once to create new objects and to mark original objects that they have
already copied, and again to remove marks.
This two-step process makes copy-object work slower than copy-seq.
If what you wish to copy is definitely a sequence,
use of {\bf copy-seq} or {\bf copy-tree} is recommended.}

\funcdesc{become}{object class}{
changes the class of {\em object} to {\em class}.
The slot structure of both the old class and the new class must be
consistent. Usually, this can be safely used only for changing
class between binary vectors, for example from an integer-vector
to a bit-vector.}

\funcdesc{replace-object}{dest src}{
dest must be an instance of the subclass of src.}

\funcdesc{class}{object}{
returns the class object of {\em object}.
To get the name of the class, send :name message to the class object.}

\funcdesc{derivedp}{object class}{
{\bf derivedp} checks if an object is instantiated from {\em class}
or {\em class}'s subclasses.
{\bf subclassp} and {\bf derivedp} functions do not search in class hierarchy:
type check always finishes within a constant time.}

\funcdesc{slot}{object class (index $|$ slot-name)}{
Returns the named or indexed slot value.}

\funcdesc{setslot}{object class (index $|$ slot-name) value}{
{\bf Setslot} is a internal function and users should not use it.
Use, instead, combination of {\bf setf} and {\bf slot}.}

\end{refdesc}

\subsection{Basic Classes}

\begin{refdesc}
\classdesc{object}{}
{}
{Object is the most basic class that is located at the top of class hierarchy.
Since it defines no slot variables, it is no use to make an instance of
{\bf object}.}

\methoddesc{:prin1}{\&optional stream \&rest mesg}{
prints the object in the standard re-readable object format,
that is, the class name and the address, enclosed by angle brackets and
preceded by a pound sign.
Any subclasses of {\bf object} can use this method to print itself with
more comprehensive information  by using
{\bf send-super} macro specifying {\em mesg} string.
An object is re-readable if it begins with \#$<$,
followed by its class name, correct address, any lisp-readable information,
and \verb~>~.
Since every data object except numbers inherits {\bf object},
you can get print forms in this notation, even for symbols or strings.
Specifying this notation, you can catch data objects that you forgot
to {\bf setq} to a symbol, as long as there happened no garbage collection
after it is printed.}

\methoddesc{:slots}{}{
returns the list of variable-name and value pair of all the slots of the
object.
You can get the value of a specific slot by applying {\bf assoc} to this
list, although you cannot alter them.}

\classdesc{propertied-object}{object}
{plist}
{defines objects that have property list.
Unlike other Common Lisp, 
EusLisp allows any objects that inherit propertied-object to have property
lists, even if they are not symbols.}

\methoddesc{:plist}{\&optional plist}{
if {\em plist} is specified, it is set to the plist slot of this object.
Previous plist, if there had been one, is lost.
Legal plist should be of the form of
\verb~((indicator1 . value1) (indicator2 . value2) ...)~.
Each \verb~indicator~ can be any lisp form that are tested its equality
with the {\bf eq} function.
When a symbol is used for an indicator, use of keyword is recommended
to ensure the equality check will be performed interpackage-widely.
{\bf :plist} returns the current plist.}

\methoddesc{:get}{indicator}{
returns the value associated with {\em indicator} in the property list.
\verb~(send x :get :y) == (cdr (assoc :y (send x :plist)))~.}

\methoddesc{:put}{indicator value}{
associates {\em value} to {\em indicator} in the plist.}

\methoddesc{:remprop}{indicator}{
removes {\em indicator} and value pair from the plist.
Further attempt to {\em :get} the value returns nil.}

\methoddesc{:name}{\&optional name}{
defines and retrieves the {\tt :name} property in the plist.
This property is used for printing.}

\methoddesc{:prin1}{\&optional stream \&rest mesg}{
prints the object in the re-readable form.
If the object has {\bf :name} property, it is printed after
the address of the object.}

\methoddesc{:slots}{}{
returns a list of variable and value pairs of this object.}

\methoddesc{:methods}{\&optional subname}{
returns a list of all methods callable by this object. If {\em subname} is given, returns only methods with names that include {\em subname}. }

\methoddesc{:get-val}{variable-name}{
returns the value of the slot designated by {\it variable-name}.
If the object does not have the variable-name slot, an error is reported.}

\methoddesc{:set-val}{variable-name value}{
sets {\it value} in the variable-name slot of this object.
If the object does not have the variable-name slot, an error is reported.}

\classdesc{metaclass}{propertied-object}
{name super cix vars types forwards methods}
{Metaclass defines classes. Classes that have own class variables
should be defined with {\bf metaclass} as their superclass.}
\methoddesc{:new}{}
{creates an instance of this class and returns it
after filling all the slots with NIL.}
\methoddesc{:super}{}{
returns the super class object of this class.
You cannot alter superclass once defclassed.}
\methoddesc{:methods}{}{
returns a list of all the methods defined in this class.
The list is composed of lists each of which describes the name of
the method, parameters, and body.}
\methoddesc{:method}{name}{
returns the method definition associated with {\em name}.
If not found, NIL is returned.}
\methoddesc{:method-names}{subname}{
returns a list of all the method names each of which contains
{\em subname} in its method name.
Methods are searched only in this class.}
\methoddesc{:all-methods}{}{
returns a list of all methods that are defined in this class and
its all the super classes.
In other words, an instance of this class can execute each of these
methods.}
\methoddesc{:all-method-names}{subname}{
returns a list of all the method names each of which matches with
{\em subname}.
The search is made from  this class up to {\bf object}.}
\methoddesc{:slots}{}{
returns the slot-name vector.}
\methoddesc{:name}{}{returns the name symbol of this class.}
\methoddesc{:cid}{}{
returns an integer that is assigned to every
instance of this class to identify its class.
This is an index to the system-internal class table, and is changed
when a new subclass is defined under this class.}
\methoddesc{:subclasses}{}{
returns a list of the direct subclass of this class.}
\methoddesc{:hierarchy}{}{
returns a list of all the subclasses defined under this class.
You can also call the {\bf class-hierarchy} function to get a comprehensive
listing of all the class hierarchy.}
\funcdesc{find-method}{object selector}{
searches for the method identified by {\em selector} in {\em object}'s
class and its super classes. This function is useful when object's
class is uncertain and you want to know whether the object can handle
the message without causing nomethod error.}
\end{refdesc}
\newpage
