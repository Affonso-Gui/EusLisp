\section{Sequences, Arrays and Tables}
\markright{\arabic{section}. Sequences, Arrays and Tables}

\subsection{General Sequences}

Vectors (one dimensional arrays) and lists are generic sequences.
A string is a sequence, since it is a vector of characters.

For the specification of result type in
{\bf map, concatenate} and {\bf coerce},
use class name symbol, such as {\tt cons, string, integer-vector, float-vector},
etc. without quotes, 
since the class object is bound to the symbol.

\begin{refdesc}

\funcdesc{elt}{sequence pos}{
{\bf elt} is the most general function to get and put (in conjunction with
{\bf setf}) value at the specific position {\em pos} in {\em sequence}.
{\em Sequence} may be a list, or a vector of arbitrary
object, bit, char, integer, or float.
{\bf Elt} cannot be applied to a multi-dimensional array.}

\funcdesc{length}{sequence}{
returns the length of {\em sequence}.
For vectors, {\bf length} finishes in constant time, but
time proportional to the length is required for a list.
{\bf Length} never terminates if {\em sequence} is a circular list.
Use {\bfx list-length}, instead.
If {\em sequence} is an array with a fill-pointer, {\bf length}
returns the fill-pointer, not the entire size of the array entity.
Use {\bf array-total-size} to know the entire size of those arrays.}

\funcdesc{subseq}{sequence start \&optional end}{
makes a copy of the subsequence from {\em start}th through  ({\em end}-1)th inclusively
out of {\em sequence}.
{\em end} is defaulted to the length of {\em sequence}.}

\funcdesc{copy-seq}{sequence}{
does shallow-copying of {\em sequence}, that is, 
only the top-level references in {\em sequence} are copied.
Use {\bf copy-tree} to copy a nested list,
or {\bf copy-object} for deep-copying of a sequence
containing recursive references.}

\funcdesc{reverse}{sequence}{
reverse the order of  {\em sequence} and returns a new sequence of the
same type as {\em sequence}.}

\funcdesc{nreverse}{sequence}{
{\bf Nreverse} is the destructive version of {\bf reverse}.
{\bf Nreverse} does not allocate memory, while {\bf reverse} does.}

\funcdesc{concatenate}{result-type \&rest sequences}{
concatenates all {\em sequences}.
Each {\em sequence} may be of any sequence type.
Unlike {\bf append}, all the sequences including the last one are copied.
{\em Result-type} should be a class such as cons, string,
vector, float-vector etc.}

\funcdesc{coerce}{sequence result-type}{
changes the type of {\em sequence}.
For examples, {\tt (coerce '(a b c) vector) = \#(a b c)}  and 
{\tt (coerce "ABC" cons) = (a b c)}.
A new sequence of type {\em result-type} is created, and each element of
{\em sequence} is copied to it.
{\em result-type} should be one of vector, integer-vector, float-vector,
bit-vector, string, cons or other user-defined classes inheriting 
one of these.
Note that {\em sequence} is copied even if its type equals to {\em result-type}.}

\funcdesc{map}{result-type function seq \&rest more-seqs}{
{\em function} is applied to a list of arguments taken from {\em seq}
and {\em more-seqs} orderly, and the result is accumulated in a sequence
of type {\em result-type}.}

\funcdesc{fill}{sequence item \&key (start 0) (end (length sequence))}{
fills {\em item} from {\em start}th through ({\em end}-1)th in {\em sequence}.
}

\funcdesc{replace}{dest source \&key start1 end1 start2 end2}{
elements in {\em dest} sequence indexed between {\em start1} and {\em end1}
are replaced with elements in {\em source} indexed between 
{\em start2} and {\em end2}.
{\em start1} and {\em start2} are defaulted to zero, and
{\em end1} and {\em end2} to the length of each sequence.
If the one of subsequences is longer than the other,
its end is truncated to match with the shorter subsequence.}

\funcdesc{sort}{sequence compare \&optional key}{
{\em sequence} is destructively sorted using Unix's quick-sort subroutine.
{\em key} is not a keyword parameter.
Be careful with the sorting of a sequence which have same elements.
For example, {\tt (sort '(1 1) \#'>)} fails because comparisons
between 1 and 1 in both direction fail.
To avoid this problem, use functions like \#'$>=$ or \#'$<=$ for comparison.}

\funcdesc{merge}{result-type seq1 seq2 pred \&key (key \#'identity)}{
two sequences {\em seq1} and {\em seq2} are merged to form a single
sequence of {\em result-type} whose elements 
satisfy the comparison specified by {\em pred}.}

\funcdesc{merge-list}{list1 list2 pred key}{
merges two lists. Unlike {\bf merge} no general sequences are allowed 
for the arguments, but {\bf merge-list} runs faster than {\bf merge}.}

\end{refdesc}

Following functions consist of one basic function and its variants
suffixed by -if and -if-not.
The basic form takes at least the item and sequence arguments,
and compares item with each element in the sequence,
and do some processing,
such as finding the index,
counting the number of appearances, removing the item, etc.
Variant forms take predicate and sequence arguments,
applies the predicate to each element of sequence, and do something
if the predicate returns non-nil (-if version), or nil (-if-not version).

\begin{refdesc}

\funcdesc{position}{item seq \&key start end test test-not key (count 1)}{
finds {\em count}th appearance of {\em item} in {\em seq} and returns
its index.
The search begins from the {\em start}th element, ignoring elements before it.
By default, the search is performed by {\bf eql}, which can be altered
by the {\em test} or {\em test-not} parameter.}

\fundesc{position-if}{predicate seq \&key start end key}

\fundesc{position-if-not}{predicate seq \&key start end key}

\funcdesc{find}{item seq \&key start end test test-not key (count 1)}{
finds {\em count}th element between the {\em start}th element
and the {\em end}th element in {\em seq}.
The element found, which is eql to {\em item} if no {\em test} or
{\em test-not} other than \#'eql is specified, is returned.}

\funcdesc{find-if}{predicate seq \&key start end key (count 1)}{
finds {\em count}th element in {\em seq} for which {\em pred}
returns non nil.}
\fundesc{find-if-not}{predicate seq \&key start end key}

\funcdesc{count}{item seq \&key start end test test-not key}{
counts the number of {\em item}s which appear between the {\em start}th element
and the {\em end}th element in {\em seq}.}

\funcdesc{count-if}{predicate seq \&key start end key}{
count the number of elements in {\em seq} for which {\em pred} returns
non nil.}
\fundesc{count-if-not}{predicate seq \&key start end key}

\funcdesc{remove}{item seq \&key start end test test-not key count}{
creates a new sequence which has eliminated {\em count} (defaulted to infinity)
occurrences of of {\em item}(s) between the {\em start}th element
and the {\em end}th element in {\em seq}.
If you are sure that there is only one occurrence of {\em item},
{\em count=1} should be specified to avoid meaningless scan over the whole
sequence.}

\fundesc{remove-if}{predicate seq \&key start end key count}
\fundesc{remove-if-not}{predicate seq \&key start end key count}
\funcdesc{remove-duplicates}{seq \&key start end key test test-not count}{
removes duplicated items in {\em seq} and creates a new sequence.}

\funcdesc{delete}{item seq \&key start end test test-not key count}{
is same with {\bf remove} except that {\bf delete} modifies {\em seq}
destructively and does not create a new sequence.
If you are sure that there is only one occurrence of {\em item},
{\em count=1} should be specified to avoid meaningless scan over the whole
sequence.}

\fundesc{delete-if}{predicate seq \&key start end key count}
\funcdesc{delete-if-not}{predicate seq \&key start end key count}{
{\em count} for {\em remove}s and {\em delete}s is defaulted to 1,000,000.
If you have a long sequence and you want to delete an element which
appears only once, :count should be specified as 1.}

\funcdesc{substitute}{newitem olditem seq \&key start end test test-not key count}{
returns a new sequence which has 
substituted the {\em count} occurrence(s)
of {\em olditem} in {\em seq} with {\em newitem}.
By default, all the {\em olditems} are substituted.}

\begin{verbatim}
(substitute #\Space #\_ "Euslisp_euslisp") ;; => "Euslisp euslisp"
\end{verbatim}

\fundesc{substitute-if}{newitem predicate seq \&key start end key count}
\fundesc{substitute-if-not}{newitem predicate seq \&key start end key count}

\funcdesc{nsubstitute}{newitem olditem seq \&key start end test test-not key count}
{substitute the {\em count} occurrences of {\em olditem} in {\em seq} with {\em newitem}
destructively. By default, all the {\em olditem}s are substituted.}
\fundesc{nsubstitute-if}{newitem predicate seq \&key start end key count}
\fundesc{nsubstitute-if-not}{newitem predicate seq \&key start end key count}

\end{refdesc}

\newpage

\subsection{Lists}

\begin{refdesc}
\funcdesc{listp}{object}{
returns T if object is an instance of cons or NIL.}
\funcdesc{consp}{object}{
equivalent to (not (atom object)). (consp '()) is nil.}
\funcdesc{car}{list}{
returns the first element in {\em list}. {\bf car} of NIL is NIL.
{\bf car} of atom is error.}
\funcdesc{cdr}{list}{returns the list which removed the first element
of {\em list}. {\bf cdr} of NIL is NIL.
{\bf cdr} of atom is error.}
\funcdesc{cadr}{list}{{\tt (cadr list) = (car (cdr list))}}
\funcdesc{cddr}{list}{{\tt (cddr list) = (cdr (cdr list))}}
\funcdesc{cdar}{list}{{\tt (cdar list) = (cdr (car list))}}
\funcdesc{caar}{list}{{\tt (caar list) = (car (car list))}}
\funcdesc{caddr}{list}{{\tt (caddr list) = (car (cdr (cdr list)))}}
\funcdesc{caadr}{list}{{\tt (caadr list) = (car (car (cdr list)))}}
\funcdesc{cadar}{list}{{\tt (cadar list) = (car (cdr (car list)))}}
\funcdesc{caaar}{list}{{\tt (caaar list) = (car (car (car list)))}}
\funcdesc{cdadr}{list}{{\tt (cdadr list) = (cdr (car (cdr list)))}}
\funcdesc{cdaar}{list}{{\tt (cdaar list) = (cdr (car (car list)))}}
\funcdesc{cdddr}{list}{{\tt (cdddr list) = (cdr (cdr (cdr list)))}}
\funcdesc{cddar}{list}{{\tt (cddar list) = (cdr (cdr (car list)))}}
\funcdesc{first}{list}{retrieves the first element in {\em list}.
{\bfx second, third, fourth, fifth, sixth, seventh, eighth} are also 
available.}
\funcdesc{nth}{count list}{
returns the {\em count}-th element in {\em list}.
Note that {\tt (nth 1 list)} is equivalent to {\tt (second list)},
and to {\tt (elt list 1)}.}

\funcdesc{nthcdr}{count list}{
applies {\bf cdr} {\em count} times to {\em list}.}

\funcdesc{last}{list}{
the last cons is returned, not the last element.}

\funcdesc{butlast}{list \&optional (n 1)}
{ returns a list which does not contain the last {\em n} elements.}

\funcdesc{cons}{car cdr}{
makes a new cons whose car is {\em car} and cdr is {\em cdr}.}

\funcdesc{list}{\&rest elements}{makes a list of {\em elements}.}

\funcdesc{list*}{\&rest elements}{
makes a list of {\em elements}, but the last element is consed in cdr:
for example, {\tt (list* 1 2 3 '(4 5)) = (1 2 3 4 5)}.}

\funcdesc{list-length}{list}{
returns the length of the {\em list}. {\em List} can be circular.}

\funcdesc{make-list}{size \&key initial-element}{
makes a list whose length is {\em size} and elements are {\em initial-element}.}

\funcdesc{rplaca}{cons a}{
replace the car of {\em cons} with a.
Use of {\bf setf} to {\bf car} is recommended.}

\funcdesc{rplacd}{cons d}{
replace the cdr of {\em cons} with d.
Use of {\bf setf} to {\bf cdr} is recommended.}

\funcdesc{memq}{item list}{
resembles {\bf member}, but test is always done by {\bf eq}.}

\funcdesc{member}{item list \&key key (test \#'eq) test-not}
{The {\em list} is searched for an element that satisfies the {\em test}.
If none is found, NIL is returned; otherwise, the tail of {\em list} beginning
with the first element that satisfied the test is returned. The {\em list}
is searched on the top level only.}

\fundesc{assq}{item alist}
\funcdesc{assoc}{item alist \&key key (test \#'eq) test-not}
{ searches the association list {\em alist}. The value returned is the
first pair in the {\em alist} such that the {\em car} of the pair satisfies
the {\em test}, or NIL if there is no such pair in the {\em alist}.}

\funcdesc{rassoc}{item alist}{
returns the first pair in {\em alist} whose cdr is equal to {\em item}.}

\funcdesc{pairlis}{l1 l2 \&optional alist}{
makes a list of pairs consing corresponding elements in {\em l1} and {\em l2}.
If {\em alist} is given, it is concatenated at the tail of the pair list
made from {\em l1} and {\em l2}.}

\funcdesc{acons}{key val alist}{
add the {\em key val} pair to {\em alist}, that is,
{\tt (cons (cons key val) alist)}.}

\funcdesc{append}{\&rest list}{
appends {\em list} to form a new list.
All the elements in {\em list}, except the last list, are copied.}

\funcdesc{nconc}{\&rest list}{
concatenates {\em list} destructively by replacing the last cdr of each
{\em list}.}

\funcdesc{subst}{new old tree}{
substitutes every {\em old} in {\em tree} with {\em new}.}

\funcdesc{flatten}{complex-list}{
{\em Complex-list} composed of atoms and lists of any depth
is transformed into a single level linear list which have all the elements
in {\em complex-list} at the top level.
For example, {\tt (flatten '(a (b (c d) e))) = (a b c d e)}}

\macrodesc{push}{item place}{
pushes item into a stack (list) bound to {\em place}.}

\macrodesc{pop}{stack}{
removes the first item from {\em stack} and returns it.
If {\em stack} is empty (nil), nil is returned.}

\macrodesc{pushnew}{item place \&key test test-not key}{
pushes {\em item} in the {\em place} list
if {\em item} is not a member of {\em place}.
The {\em test}, {\em test-not} and {\em key} arguments are
passed to the member function.}

\funcdesc{adjoin}{item list}{
The item is added at the head of the list if it is not included
in the list.}

\funcdesc{union}{list1 list2 \&key (test \#'eq) test-not (key \#'identity)}{
returns union set of two lists.}
\funcdesc{subsetp}{list1 list2 \&key (test \#'eq) test-not (key \#'identity)}
{tests if {\em list1} is a subset of {\em list2}, i.e. if each element
of {\em list1} is a member of {\em list2}.}

\funcdesc{intersection}{list1 list2 \&key (test \#'eq) test-not (key \#'identity)}{
returns the intersection of two sets, {\em list1} and {\em list2}.}

\funcdesc{set-difference}{list1 list2 \&key (test \#'eq) test-not (key \#'identity)}{
returns the list whose elements are only contained in {\em list1}
and not in {\em list2}.}
\funcdesc{set-exclusive-or}{list1 list2 \&key (test \#'eq) test-not (key \#'identity)}{
returns the list of elements that appear only either in {\em list1} or {\em list2}.}
\funcdesc{list-insert}{item pos list}{
Insert {\em item} as the {\em pos}'th element in {\em list} destructively.
If {\em pos} is bigger than the length of {\em list}, {\em item} is
nconc'ed at the tail.
For example, {\tt (list-insert 'x 2 '(a b c d)) = (a b x c d)}}

\funcdesc{copy-tree}{tree}{
returns the copy of {\em tree} which may be a nested list
but cannot have circular reference. Circular lists can be copied by
{\bf copy-object}.
Actually, {\bf copy-tree} is simply coded as {\tt (subst t t tree)}.}

\funcdesc{mapc}{func arg-list \&rest more-arg-lists}{
applies {\em func} to a list of N-th elements in {\em arg-list} and each of
{\em more-arg-lists}.
The results of application are ignored and {\em arg-list} is returned.}

\funcdesc{mapcar}{func \&rest arg-list}{
maps {\em func} to each element of {\em arg-list},
and makes a list from all the results.
Before using {\bf mapcar}, try {\bf dolist}.}

\funcdesc{mapcan}{func arg-list \&rest more-arg-lists}{
maps {\em func} to each element of {\em arg-list},
and makes a list from all the results by {\bf nconc}.
{\bf Mapcan} is suitable for filtering (selecting) elements
in {\em arg-list}, since nconc does nothing with NIL.}

\end{refdesc}
\newpage

\subsection{Vectors and Arrays}

Up to seven dimensional arrays are allowed.
A one-dimensional array is called vector.
Vectors and lists are grouped as sequence.
If the elements of an array is of any type, the array is said to be general.
If an array does not have fill-pointer, is not displaced to
another array, or is adjustable, the array is said to be simple.

Every array element can be recalled by {\bf aref} and set by {\bf setf} in
conjunction with aref.
But for simple vectors, there are simpler and faster access functions:
{\bf svref} for simple general vectors, {\bf char} and {\bf schar} for
simple character vectors (string), {\bf bit} and {\bf sbit} for
simple bit vectors.  When these functions are compiled,
the access is expanded in-line and no type check and boundary check are 
performed.

Since a vector is also an object,
it can be made by instantiating some vector-class.
There are five kinds of built-in vector-classes;
vector, string, float-vector, integer-vector and bit-vector.
In order to ease instantiation of vectors, the function make-array
is provided.
Element-type should be one of {\bf :integer, :bit, :character, :float, :foreign}
or user-defined vector class.
{\bf :initial-element} and {\bf :initial-contents} key word arguments are
available to set initial values of the array you make.

\begin{refdesc}

\constdesc{array-rank-limit}{7}
\constdesc{array-dimension-limit}{\#x1fffffff, logically, but stricter
limit is imposed by the physical or virtual memory size of the system.}

\funcdesc{vectorp}{object}{
An array is not a vector even if it is one dimensional.
T is returned for vectors, integer-vectors, float-vectors, strings,
bit-vectors or other user-defined vectors.}
\funcdesc{vector}{\&rest elements}{
makes a simple vector from {\em elements}.}

\longdescription{make-array}{
dims \&key \= (element-type vector) \` [function] \\
\> (initial-contents nil) \\
\> (initial-element nil) \\
\> (fill-pointer nil)  \\
\> (displaced-to nil) \\
\> (displaced-index-offset 0) \\
\> (adjustable nil)}
{makes a vector or array. 
{\em dims} is either an integer or a list.
If {\em dims} is an integer, a simple-vector is created.}

\funcdesc{svref}{vector pos}{
returns {\em pos}th element of {\em vector}.
{\em Vector} must be a simple general vector.}

\funcdesc{aref}{vector \&rest indices}{
returns the element indexed by {\em indices}.
{\bf Aref} is not very efficient
because it needs to dispatch according to the type of {\em vector}.
Type declarations should be given to improve the speed of compiled code
whenever possible.}

\funcdesc{vector-push}{val array}{
store {\em val} at the {\em fill-pointer}th slot in {\em array}.
{\em array} must have a {\em fill-pointer}.
After val is stored,
the fill-pointer is advanced by one to point to the next location.
If it exceeds the array boundary, an error is reported.}

\funcdesc{vector-push-extend}{val array}{
Similar to  {\bf vector-push} except that
the size of the array is automatically extended
when {\em array}'s fill-pointer reaches  the end.}

\funcdesc{arrayp}{obj}
{T if {\em obj} is an instance of array or vector.}

\funcdesc{array-total-size}{array}{
returns the total number of elements of {\em array}.}
\fundesc{fill-pointer}{array}
\fundesc{array-rank}{array}
\funcdesc{array-dimensions}{array}{
returns a list of array-dimensions.}

\funcdesc{array-dimension}{array axis}{
Axis starts from 0. {\bf array-dimension} returns the {\em axis}th
dimension of {\em array}.}

\funcdesc{bit}{bitvec index}{
returns the {\em index}th element of {\em bitvec}.
Use {\bf setf} and {\bf bit} to change an element of a bit-vector.}
\fundesc{bit-and}{bits1 bits2 \&optional result}
\fundesc{bit-ior}{bits1 bits2 \&optional result}
\fundesc{bit-xor}{bits1 bits2 \&optional result}
\fundesc{bit-eqv}{bits1 bits2 \&optional result}
\fundesc{bit-nand}{bits1 bits2 \&optional result}
\fundesc{bit-nor}{bits1 bits2 \&optional result}
\funcdesc{bit-not}{bits1 \&optional result}{
For bit vectors {\em bits1} and {\em bits2} of the same length,
their boolean and, inclusive-or,
exclusive-or, equivalence, not-and, not-or and not are returned, respectively.}

\end{refdesc}

\newpage

\subsection{Characters and Strings}

There is no character type in EusLisp;
a character is represented by an integer.
In order to handle strings representing file names,
use {\bf pathname}s described in \ref{Pathnames}.

\begin{refdesc}

\funcdesc{digit-char-p}{ch}{
T if {\em ch} is \#$\backslash$0 through \#$\backslash$9.}

\funcdesc{alpha-char-p}{ch}{
T if {\em ch} is \#$\backslash$A through \#$\backslash$Z or
\#$\backslash$a through \#$\backslash$z.}

\funcdesc{upper-case-p}{ch}{
T if {\em ch} is \#$\backslash$A through \#$\backslash$Z.}

\funcdesc{lower-case-p}{ch}{
T if {\em ch} is  \#$\backslash$a through \#$\backslash$z.}

\funcdesc{alphanumeric-p}{ch}{
T if {\em ch} is  \#$\backslash$0 through \#$\backslash$9,
\#$\backslash$A through \#$\backslash$Z
or \#$\backslash$a through \#$\backslash$z.}

\funcdesc{char-upcase}{ch}{
convert the case of {\em ch} to upper.}

\funcdesc{char-downcase}{ch}{
convert the case of {\em ch} to lower.}

\funcdesc{char}{string index}{
returns {\em index}th character in {\em string}.}

\funcdesc{schar}{string index}{
extracts a character from {\em string}. Use {\bf schar} only if the
type of {\em string} is definitely known and no type check is required.}

\funcdesc{stringp}{object}{
{\em string} is a vector of bytes (integers less than 256).}

\funcdesc{string-upcase}{str \&key start end}{
converts {\em str} to upper case string and returns a new string.}

\funcdesc{string-downcase}{str \&key start end}{
converts {\em str} to lower case string and returns a new string.}

\funcdesc{nstring-upcase}{str}{
converts {\em str} to upper case string destructively.}

\funcdesc{nstring-downcase}{str \&key start end}{
converts {\em str} to lower case string destructively.}

\funcdesc{string=}{str1 str2 \&key start1 end1 start2 end2}{
T if {\em str1} is equal to {\em str2}.
{\em string=} is case sensitive.}

\funcdesc{string-equal}{str1 str2 \&key start1 end1 start2 end2}{
tests equality of {\em str1} and {\em str2}.
{\bf string-equal} is not case sensitive.}

\funcdesc{string}{object}{
gets string notation of {\em object}. If {\em object} is a string, the {\em object}
is returned. If {\em object} is a symbol, its pname is copied and returned.
Note that (equal (string 'a) (symbol-pname 'a))==T, but
(eq (string 'a) (symbol-pname 'a))==NIL.
If object is number its string representation is returned
(this is incompatible with Common Lisp).
In order to get string representation for more complex objects,
use {\bf format} with NIL in the first argument.}

\fundesc{string$<$}{str1 str2}
\fundesc{string$<=$}{str1 str2}
\fundesc{string$>$}{str1 str2}
\fundesc{string$>=$}{str1 str2}

\fundesc{string-left-trim}{bag str}
\funcdesc{string-right-trim}{bag str}{
{\em str} is scanned from the left(or right),
and its elements are removed if
it is included in the {\em bag} list.
Once a character other than the ones in the {\em bag} is found,
further scan is aborted and the rest of {\em str}
is returned.}

\funcdesc{string-trim}{bag str}{
{\em Bag} is a sequence of character codes.
A new copy of {\em str} which does not contain characters specified in {\em bag}
in its both end is made and returned.}

\funcdesc{substringp}{sub string}{
T if string {\em sub} is contained in {\em string} as a substring.
Not case sensitive.}

\end{refdesc}

\subsection{Foreign Strings}
A foreign-string is a kind of byte-vector whose entity is held somewhere
outside EusLisp's heap.
While a normal string is represented by a sequence of bytes and its length,
a foreign-string holds the length and the address of the string entity.
Although foreign-string is a string,
some string and  sequence functions cannot be applicable.
Only {\bf length}, {\bf aref}, {\bf replace}, {\bf subseq} and {\bf copy-seq}
recognize the foreign-string, 
and application of other functions may cause a crash.

A foreign-string may refer to a part of I/O space usually
taken in /dev/a??d?? special file where ?? is either 32 or 16.
In case the device attached in one of these I/O space only responds
to byte access, {\bf replace} always copies element byte by byte,
which is relatively slow when a large chunk of memory is accessed
consecutively.

\begin{refdesc}
\funcdesc{make-foreign-string}{address length}{
makes an instance of foreign-string located at {\em address} 
and spanning for {\em length} bytes.
For example, {\tt (make-foreign-string (unix:malloc 32) 32)}
makes a reference to a 32-byte memory located outside EusLisp's heap.}

\end{refdesc}
\newpage

\subsection{Hash Tables}

Hash-table is a class to search for the value associated with a key,
as accomplished by {\bf assoc}.
For a relatively large problem,
hash-table performs better than assoc, since time required for searching remains constant even
the number of key-value pairs increases.
Roughly speaking, hash-table should be used in search spaces with
more than 100 elements, and assoc in smaller spaces.

Hash-tables automatically expands if the number of elements
in the table exceeds rehash-size.
By default, expansion occurs when a half of the table is filled.
{\bf sxhash} function returns a hash value which is independent
of memory address of an object, and hash values for {\bf equal} objects
are always the same.
So, hash tables can be re-loadable since they use sxhash as their default
hashing functions.
While sxhash is robust and safe,
it is relatively slow because it scans all the elements 
in a sequence or a tree.
For faster hashing, you may choose another hash function appropriate
for your application.
To change the hash function, send {\tt :hash-function} message
to the hash-table.
In simple cases, it is useful to change hash function from \#'{\bf sxhash}
to \#'{\bf sys:address}.
This is possible because the addresses of any objects
never change in a EusLisp process.

\begin{refdesc}

\funcdesc{sxhash}{obj}{
calculates the hash value for {\em obj}.
Two objects which are {\bf equal} are guaranteed to yield
the same hash value.
For a symbol, hash value for its pname is returned.
For numbers, their integer representations are returned.
For a list, sum of hash values for all its elements is returned.
For a string, shifted sum of each character code is returned.
For any other objects, {\bf sxhash} is recursively called to calculate
the hash value of each slot, and the sum of them is returned.}

\funcdesc{make-hash-table}{\&key (size 30) (test \#'eq) (rehash-size 2.0)}{
creates a hash table and returns it.}


\funcdesc{gethash}{key htab}{
gets the value that corresponds to {\em key} in {\em htab}.
{\bf Gethash} is also used to set a value to key by combining with {\bf setf}.
When a new entry is entered in a hash table, and the number of filled slots
in the table exceeds 1/rehash-size, then the hash table is automatically
expanded to twice larger size.}

\funcdesc{remhash}{key htab}{
removes a hash entry designated by {\em key} in {\em htab}.}

\funcdesc{maphash}{function htab}{
maps {\em function} to all the elements of {\em htab}.}

\funcdesc{hash-table-p}{x}{
T if {\em x} is an instance of class hash-table.}

\classdesc{hash-table}{object}
{(key value count \\
\> hash-function test-function \\
\> rehash-size empty deleted)}{
defines hash table.
{\em Key} and {\em value} are simple-vectors of the same {\em size}.
{\em Count} is the number of filled slots in {\em key} and {\em value}.
{\em Hash-function} is defaulted to {\bf sxhash} and
{\em test-function} to {\bf eq}.
{\em Empty} and {\em deleted} are uninterned symbols to indicate
slots are empty or deleted in the {\em key} vector.}

\methoddesc{:hash-function}{newhash}{
changes the hash function of this hash table to {\em newhash}.
{\em Newhash} must be a function with one argument and returns an integer.
One of candidates for {\em newhash} is {\bf system:address}.}

\end{refdesc}

\subsection{Queue}

A queue is a data structure that allows insertion and retrieval of data
in the FIFO manner, i.e. the first-in first-out order.
Since the queue class is defined by extending the cons class,
ordinary list functions can be applied to a queue.
For example, caar retrieves the next element to be dequeued,
and cadr gets the element that is queued most recently.

\begin{refdesc}

\classdesc{queue}{cons}{(car cdr)}{
defines queue (FIFO) objects.}

\methoddesc{:init}{}{
initializes the queue to have no elements.}

\methoddesc{:enqueue}{val}{
puts val in the queue as the most recent element.}

\methoddesc{:dequeue}{\&optional (error-p nil)}{
retrieves the oldest value in the queue, and removes it of the queue.
If the queue is empty, it reports an error when error-p is non-nil,
or returns NIL otherwise.}

\methoddesc{:empty?}{}{returns T if the queue is empty.}
\methoddesc{:length}{}{returns the length of the queue.}

\methoddesc{:trim}{s}{discard old entries to keep the size of this queue to s.}
\methoddesc{:search}{item \&optional (test \#'equal)}{find element which
  is equal to item. the search is performed by equal, which can be altered by test}
\methoddesc{:delete}{item \&optional (test \#'equal) (count
  1)}{eliminate count occurrences of item in this queue.}
\methoddesc{:first}{}{returns the first entry (oldest value) of this queue.}
\methoddesc{:last}{}{returns tha last entry (newest value) of this queue.}

\end{refdesc}




\newpage

